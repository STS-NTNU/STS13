\section{Feature generation methods}
\label{sec:methods}

The methods used for creating new features utilise soft cardinality and character n-grams.
One set of features
%, based only on surface text information, 
were extracted using
soft cardinality \cite{chavez_text_2010}, which has been used successfully for the STS
task in previous SemEval editions \cite{jimenez_soft_2012,jimenez_softcardinality_core:_2013}. 
The NTNU systems utilise an ensemble of 18 such measures with different similarity functions,
as further described in Section~\ref{sec:softcard}.

Section~\ref{subrep-features} then introduces the similarity measures based on character 
n-gram feature representations. 
They replace character n-gram features with cluster frequencies or vector values 
based on the n-gram collocational structure learned in an unsupervised manner from text data. 
%Three different techniques are used to induce such representations: 
%Brown clustering \cite{brown1992class}, log linear skip-gram representations \cite{mikolov2013efficient}, 
%and Latent Semantic Indexing (LSI) topic vectors \cite{deerwester1990indexing}. 
A variety of n-gram feature representations were trained on subsets of Wikipedia and the best performing ones 
were used for the new  measures, which are based on cosine similarity between the document vectors 
derived from each sentence in a given pair.


%%% Local Variables: 
%%% mode: latex
%%% TeX-master: "sts14-ntnu"
%%% End: 
