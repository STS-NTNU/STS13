\section{Feature and parameter optimisation}
\label{sec:optimisation}

The extracted features and the parameters for the two methods described
in the previous section were optimised over several sets of training data.
As no training data was explicitly provided for the STS evaluation campaign
this year, we used different training sets from past campaigns and from
Wikipedia for the new test sets.

\subsection{Training data and pre-processing}
\label{preprocessing}

The selected training-test sets pairs are shown in Table~\ref{tab:training-test-sets}.
These were used for optimising the parameters of the soft cardinality methods.
The character n-gram representation vectors were trained in an unsupervised manner on subsets of 
Wikipedia consisting of the first 12 million words ($10^8$ characters) referred 
to as {\it Wiki8} and 125 million words ($10^9$ characters; {\it Wiki9}).

\begin{table}[t!]
\begin{centering}
\begin{tabular}{|l|l|}
\hline 
\multicolumn{1}{|c|}{\bf Test set} & \multicolumn{1}{c|}{\bf Training set}\tabularnewline
\hline 
{ OnWN} & { OnWN 2012 and 2013 test}\tabularnewline
\hline 
{ headlines} & { headlines 2013 test}\tabularnewline
\hline 
{ images} & { MSRvid 2012 train + test}\tabularnewline
\hline 
{ deft-news} & { MSRvid 2012 train + test}\tabularnewline
\hline 
\multirow{2}{*}{{ deft-forum}} & { MSRvid 2012 train and test +}\tabularnewline
 & { OnWN 2012 and 2013 test}\tabularnewline
\hline 
\multirow{2}{*}{{ tweet-news}} & { SMTeuroparl 2012 test +}\tabularnewline
 & { SMTnews 2012 test }\tabularnewline
\hline 
\end{tabular}
\par\end{centering}

\centering{}\caption{Training-test set pairs\label{tab:training-test-sets}}
\end{table}

First, however, the training data had to be pre-processed.
Hence, all texts in Table~\ref{tab:training-test-sets} were passed through
\emph{i)} tokenization and stop-word removal (provided by NLTK, \newcite{bird2009natural}),
%\footnote{http://www.nltk.org/ 
\emph{ii)} conversion to lowercase characters, 
\emph{iii)} punctuation and special character removal (e.g., ``.'', ``;'', ``\$'', ``\&''),
and \emph{iv)} Porter stemming.
The output was used to obtain \emph{idf} weights and the soft cardinality features.

The Wikipedia texts were only pre-processed by 
\emph{i)} removal of punctuation and extra whitespace, 
\emph{ii)} replacing numbers with their single digit word (`one', `two', etc.), 
and \emph{iii)} lowercasing all text. 
Character n-grams including whitespace were then generated from these texts. 

\subsection{Soft cardinality parameter optimisation}
\label{softcard-optimisation}

The first feature  in Table~\ref{tab:features}, $\mathbf{STS_{sim}}$, was used
to optimise the four parameters $\alpha$, $\beta$, $bias$, and $p$ in the following way.
First, we built a text similarity function reusing Eq.~\ref{eq:symm_tversky}
for comparing two sets of words (instead of two sets of character
3-grams) and replacing the classic cardinality $|*|$ by the soft
cardinality $|*|_{sim}$ from Eq.~\ref{eq:soft_card}. This text similarity
function adds three parameters ($\alpha'$, $\beta'$, and $bias'$) to
the initial set of four parameters.
% $\alpha$, $\beta$, $bias$ and $p$.

Second, these seven parameters were set to their default values and the
scores obtained from this function for each pair of sentences was compared
to the gold standard in the training data using Pearson's correlation.
The parameter search space was then explored iteratively using
hill-climbing until reaching optimal Pearson's correlation. 
The criteria for assignment of training-test
sets pairs was by closeness of average character length.
The optimal training parameters are shown in Table~\ref{tab:Optimal-parameters}.

\begin{table}[t!]
\begin{tabular}{|l|ccccccc|}
\hline 
{\scriptsize\bf Data} & {\footnotesize $\alpha$} & {\footnotesize $\beta$} & {\footnotesize $bias$} & {\footnotesize $p$} & {\footnotesize $\alpha'$} & {\footnotesize $\beta$} & {\footnotesize $bias'$}\tabularnewline
\hline 
{\scriptsize OnWN} & {\scriptsize 0.53} & {\scriptsize -0.53} & {\scriptsize 1.01} & {\scriptsize 1.00} & {\scriptsize -4.89} & {\scriptsize 0.52} & {\scriptsize 0.46}\tabularnewline
{\scriptsize headlines} & {\scriptsize 0.36} & {\scriptsize -0.29} & {\scriptsize 4.17} & {\scriptsize 0.85} & {\scriptsize -4.50} & {\scriptsize 0.43} & {\scriptsize 0.19}\tabularnewline
{\scriptsize images} & {\scriptsize 1.12} & {\scriptsize -1.11} & {\scriptsize 0.93} & {\scriptsize 0.64} & {\scriptsize -0.98} & {\scriptsize 0.50} & {\scriptsize 0.11}\tabularnewline
{\scriptsize deft-news} & {\scriptsize 3.36} & {\scriptsize -0.64} & {\scriptsize 1.37} & {\scriptsize 0.44} & {\scriptsize 2.36} & {\scriptsize 0.72} & {\scriptsize 0.02}\tabularnewline
{\scriptsize deft-forum} & {\scriptsize 1.01} & {\scriptsize -1.01} & {\scriptsize 0.24} & {\scriptsize 0.93} & {\scriptsize -2.71} & {\scriptsize 0.42} & {\scriptsize 1.63}\tabularnewline
{\scriptsize tweet-news} & {\scriptsize 0.13} & {\scriptsize 0.14} & {\scriptsize 2.80} & {\scriptsize 0.01} & {\scriptsize 2.66} & {\scriptsize 1.74} & {\scriptsize 0.45}\tabularnewline
\hline 
\end{tabular}\caption{Optimal parameters used for each dataset\label{tab:Optimal-parameters}}
\end{table}

\subsection{Parameters for n-gram feature training}
\label{ngram-optimisation}

The character n-gram feature representation vectors were trained varying parameters of 
n-gram size, cluster size, and term frequency cut-offs for all models.
For log linear skip-gram models our intuition is that a larger skip-gram context is needed than the 5 or 10 wide skip-grams 
used to train word-based representations due to the smaller term vocabulary and dependency between adjacent n-grams,
so instead we trained models using skip-gram widths of 25 or 50 terms. 
Term frequency cut-offs were set to limit the model size, but also potentially serve as a regularization on the resulting measure.
%
In detail, the following sublexical representation measures are used:

\begin{itemize}
\item Log linear skip-gram representations of character 3- and 4-grams of size 1000 and 2000, resp. 
Trained on the Wiki8 corpus using a skip gram window of size 25 and 50, and frequency cut-off of 5 .
\item Brown clusters with size 1024 of character 4-grams using a frequency cut-off of 20.
\item Brown clusters of character 3-, 4- and 5-grams with cluster sizes of resp. 1024, 2048 and 1024.
The representations are trained on the Wiki9 corpus with successively increasing frequency cut-offs of 20, 320 and 1200.
\item LSI topic vectors based on character 4-grams of size 2000.   Trained on the Wiki8 corpus using a frequency cut-off of 5.
\item LSI topic vectors based on character 4-grams of size 1000. Trained on the Wiki9 corpus using a frequency cut-off of 80.
\end{itemize}

%%% Local Variables: 
%%% mode: latex
%%% TeX-master: "sts14-ntnu"
%%% End: 
