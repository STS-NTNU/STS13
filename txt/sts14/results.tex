
\section{Results}
\label{sec:results}

\begin{table*}
  \centering
 \begin{tabular*}{\linewidth}{@{\extracolsep{\fill}}lrcrcrcc}
& \multicolumn{2}{c}{\bf NTNU-run1} & \multicolumn{2}{c}{\bf NTNU-run2} &  \multicolumn{2}{c}{\bf NTNU-run3} & {\bf  Best} \\
{\bf Data} & \multicolumn{1}{c}{$r$} & \multicolumn{1}{c}{\footnotesize rank} & \multicolumn{1}{c}{$r$} 
& \multicolumn{1}{c}{\footnotesize rank} & \multicolumn{1}{c}{$r$} & \multicolumn{1}{c}{\footnotesize rank} & {$r$} \\
% \begin{tabular}{|l|c|c|c|c|}
%    \hline
%   Category & NTNU-run1 & NTNU-run2 & NTNU-run3 & Best \\
%    \hline
%    Rank & 21 & 4 & 3 & \\
    \hline
    deft-forum & 0.4369 & \rank{16} & 0.5084 & \rank{2} & 0.5305 & \rank{1} & 0.5305 \\
    deft-news & 0.7138 & \rank{14} & 0.7656 & \rank{6} & 0.7813 & \rank{2} & 0.7850 \\
    headlines & 0.7219 & \rank{17} & 0.7525 & \rank{13} & 0.7837 & \rank{1} & 0.7837 \\
    images & 0.8000 & \rank{9} & 0.8129 & \rank{4} & 0.8343 & \rank{1} & 0.8343 \\
    OnWN & 0.8348 & \rank{7} & 0.7767 & \rank{20} & 0.8502 & \rank{4} & 0.8745 \\
    tweet-news & 0.4109 & \rank{33} & 0.7921 & \rank{1} & 0.6755	& \rank{13} & 0.7921 \\
    \hline
    mean & 0.6531 & \rank{20} & 0.7347 & \rank{4} & 0.7426 & \rank{2} & 0.7429 \\
    weighted mean & 0.6631 & \rank{21} & 0.7491 & \rank{4} & 0.7549 & \rank{3} & 0.7610 \\
    \hline
  \end{tabular*}
  \caption{Final evaluation results for the submitted systems.}
  \label{tab:results}
\end{table*}

The final evaluation results for the three submitted systems are shown in Table~\ref{tab:results}. 
The systems using sublexical representation based measures show competitive performance, 
ranking third and fourth among the submitted systems with a weighted mean correlation of ${\sim}0.75$. 
They also produced the best result in four out of the six text categories in the evaluation dataset,
with NTNU-run3 being the \#1 system on deft-forum, headlines and images, \#2 on deft-news, and \#4 on OnWN.
It would thus have been the clear winner if it had not been for its sub-par performance on the tweet-news
dataset, which on the other hand is the category NTNU-run2 was the best of all systems on. 

The system based solely on soft cardinality features, NTNU-run1 displays more modest performance
ranking at $21^{\rm st}$ place (of the in total 38 submitted systems) with ${\sim}0.66$ correlation. 

From the results it can be noted that for most categories the sublexical representation measures show
strong performance in NTNU-run2, with a significantly better result for the combined system NTNU-run3. 
This indicates that while the soft cardinality features are weaker predictors overall, they are complimentary 
to the sublexical and lexical features of NTNU-run2. 
It is also indicative that this is not the case for the tweet-news category, where the text is more ``free form'' 
and less normative, so it would be expected that sublexical approaches should have stronger performance.

%%% Local Variables: 
%%% mode: latex
%%% TeX-master: "sts14-ntnu"
%%% End: 
