
\section{Soft cardinality measures}
\label{sec:softcard}

Soft cardinality resembles classical set cardinality as it is a method
for counting the number of elements in a set, but differs from it
in that similarities among elements are being considered for
the ``soft counting''. 
The soft cardinality of a set of words 
$A=\{a_{1},a_{2},..,a_{|A|}\}$  (a sentence) is defined by:

\begin{equation}
|A|_{sim}=\sum_{i=1}^{|A|}\frac{w_{a_{i}}}{\sum_{j=1}^{|A|}sim(a_{i},a_{j})^{p}}\label{eq:soft_card}
\end{equation}

\noindent
Where $p$ is a parameter that controls the cardinality's softness ($p$'s default value is 1)
and $w_{a_{i}}$ are weights for each word,
obtained through inverse document frequency (\emph{idf}) weighting.
$sim(a_{i},a_{j})$ is a similarity function that compares two words $a_{i}$ and $a_{j}$
using the symmetrized Tversky's index \cite{tversky_features_1977,jimenez_softcardinality_core:_2013}
representing them as sets of 3-grams of characters. 
That is, $a_{i}=\{a_{i,1},a_{i,2},...,a_{i,|a_{i}|}\}$
where $a_{i,n}$ is the $n$-th character trigram in the word $a_{i}$ in $A$. 
Thus, the proposed word-to-word similarity is given by:

\begin{equation}
sim(a_{i},a_{j}) = \frac{|c|}{\beta(\alpha|a_{min}|+(1-\alpha)|a_{max}|)+|c|}\label{eq:symm_tversky}
\end{equation}
\begin{equation*}
\begin{cases}
|c| &= |a_{i}\cap a_{j}|+bias_{sim}\\
|a_{min}| &= \min{|a_{i}\setminus a_{j}|,|a_{j}\setminus a_{i}|}\\
|a_{max}| &= \max{|a_{i}\setminus a_{j}|,|a_{j}\setminus a_{i}}
\end{cases}
\end{equation*}

The $sim$ function is equivalent to the Dice's coefficient
if the parameters are given their default values:
$\alpha=0.5$, $\beta=1$ and $bias=0$.

The soft cardinalities of any pair of sentences $A$, $B$
and $A\cup B$ can be obtained using Eq.~\ref{eq:soft_card}. 
The soft cardinality of the intersection is approximated by 
$|A\cap B|_{sim}=|A|_{sim}+|B|_{sim}-|A\cup B|$.
These four basic soft cardinalities are algebraically recombined to produce
an extended set of 18 features as shown in Table~\ref{tab:features}. 

The first feature, $\mathbf{STS_{sim}}$, was used to optimize the four
parameters $\alpha$, $\beta$, $bias$, and $p$ in the following way.
First, we build a text similarity function reusing Eq.~\ref{eq:symm_tversky}
for comparing two sets of words (instead of two sets of character
3-grams) and replacing the classic cardinality $|*|$ by the soft
cardinality $|*|_{sim}$ from Eq.~\ref{eq:soft_card}. This text similarity
function adds three parameters ($\alpha'$, $\beta'$, and $bias'$) to
the initial set of four parameters $\alpha$, $\beta$, $bias$ and $p$.
Second, these seven parameters were set to their default values and the
scores obtained from this function for each pair of sentences was compared
to the gold standard in the training data using Pearson's correlation.
Finally, iteratively, the parameter search space was explored using
hill-climbing until optimal Pearson's correlation was reached. As no
training data was explicitly provided for the STS evaluation campaign
this year, we used different training sets from past campaigns for
the new test sets. The criteria for the assignment of training-test
sets pairs was by their closeness of average character length.
The selected training-test sets pairs are shown in Table~\ref{tab:training-test-sets}
and the optimal training parameters in Table~\ref{tab:Optimal-parameters}.

Before using extracting the proposed features, all texts were preprocessed
by: \emph{i)} tokenization and stop-word removal (provided by NLTK%
\footnote{http://www.nltk.org/%
}), \emph{ii)} conversion to lowercase characters, \emph{iii)} punctuation
and special character removal (e.g. ``.'', ``;'', ``\$'', ``\&''),
and \emph{iv) }stemming using Porter's algorithm. Next, this preprocessed
output is used to obtain the \emph{idf} weights and the features described
above.

\begin{table}[t!]
\begin{centering}
\begin{tabular}{|c|c|c|c|}
\hline 
$\mathbf{STS_{sim}}$  & $\nicefrac{|A|-|A\cap B|}{|A|}$\tabularnewline
\hline 
{\small $|A|$} &  $\nicefrac{|A|-|A\cap B|}{|A\cup B|}$\tabularnewline
\hline 
{\small $|B|$} &  $\nicefrac{|B|}{|A\cup B|}$\tabularnewline
\hline 
{\small $|A\cap B|$} & $\nicefrac{|B|-|A\cap B|}{|B|}$\tabularnewline
\hline 
{\small $|A\cup B|$} &  $\nicefrac{|B|-|A\cap B|}{|A\cup B|}$\tabularnewline
\hline 
{\small $|A|-|A\cap B|$} & $\nicefrac{|A\cap B|}{|A|}$\tabularnewline
\hline 
{\small $|B|-|A\cap B|$}  & $\nicefrac{|A\cap B|}{|B|}$\tabularnewline
\hline 
{\small $|A\cap B|-|A\cap B|$} & $\nicefrac{|A\cap B|}{|A\cup B|}$\tabularnewline
\hline 
$\nicefrac{|A|}{|A\cup B|}$ & $\nicefrac{|A\cup B|-|A\cap B|}{|A\cap B|}$\tabularnewline
\hline
\end{tabular}
\\[1ex]
{\hfill\em\footnotesize NB: in this table only, $|*|$ is short for $|*|_{sim}$\hfill}
\end{centering}
\caption{Soft cardinality features\label{tab:features}}
\end{table}


\begin{table}[t!]
\begin{centering}
\begin{tabular}{|l|l|}
\hline 
\multicolumn{1}{|c|}{\bf\scriptsize Test set} & \multicolumn{1}{c|}{\bf\scriptsize Training set}\tabularnewline
\hline 
{\scriptsize OnWN} & {\scriptsize OnWN 2012 and 2013 test}\tabularnewline
\hline 
{\scriptsize headlines} & {\scriptsize headlines 2013 test}\tabularnewline
\hline 
{\scriptsize images} & {\scriptsize MSRvid 2012 train + test}\tabularnewline
\hline 
{\scriptsize deft-news} & {\scriptsize MSRvid 2012 train + test}\tabularnewline
\hline 
\multirow{2}{*}{{\scriptsize deft-forum}} & {\scriptsize MSRvid 2012 train and test +}\tabularnewline
 & {\scriptsize OnWN 2012 and 2013 test}\tabularnewline
\hline 
\multirow{2}{*}{{\scriptsize tweet-news}} & {\scriptsize SMTeuroparl 2012 test +}\tabularnewline
 & {\scriptsize SMTnews 2012 test }\tabularnewline
\hline 
\end{tabular}
\par\end{centering}

\centering{}\caption{Training-test sets pairs\label{tab:training-test-sets}}
\end{table}


\begin{table}[t!]
\begin{tabular}{|l|ccccccc|}
\hline 
{\footnotesize\bf Data} & {\footnotesize $\alpha$} & {\footnotesize $\beta$} & {\footnotesize $bias$} & {\footnotesize $p$} & {\footnotesize $\alpha'$} & {\footnotesize $\beta$} & {\footnotesize $bias'$}\tabularnewline
\hline 
{\scriptsize OnWN} & {\scriptsize 0.53} & {\scriptsize -0.53} & {\scriptsize 1.01} & {\scriptsize 1.00} & {\scriptsize -4.89} & {\scriptsize 0.52} & {\scriptsize 0.46}\tabularnewline
{\scriptsize headlines} & {\scriptsize 0.36} & {\scriptsize -0.29} & {\scriptsize 4.17} & {\scriptsize 0.85} & {\scriptsize -4.50} & {\scriptsize 0.43} & {\scriptsize 0.19}\tabularnewline
{\scriptsize images} & {\scriptsize 1.12} & {\scriptsize -1.11} & {\scriptsize 0.93} & {\scriptsize 0.64} & {\scriptsize -0.98} & {\scriptsize 0.50} & {\scriptsize 0.11}\tabularnewline
{\scriptsize deft-news} & {\scriptsize 3.36} & {\scriptsize -0.64} & {\scriptsize 1.37} & {\scriptsize 0.44} & {\scriptsize 2.36} & {\scriptsize 0.72} & {\scriptsize 0.02}\tabularnewline
{\scriptsize deft-forum} & {\scriptsize 1.01} & {\scriptsize -1.01} & {\scriptsize 0.24} & {\scriptsize 0.93} & {\scriptsize -2.71} & {\scriptsize 0.42} & {\scriptsize 1.63}\tabularnewline
{\scriptsize tweet-news} & {\scriptsize 0.13} & {\scriptsize 0.14} & {\scriptsize 2.80} & {\scriptsize 0.01} & {\scriptsize 2.66} & {\scriptsize 1.74} & {\scriptsize 0.45}\tabularnewline
\hline 
\end{tabular}\caption{Optimal parameters used for each dataset\label{tab:Optimal-parameters}}
\end{table}
This method for obtaining features from pairs of texts was also used
successfully in other SemEval tasks such as 
\emph{cross-lingual textual entailment }and \emph{student response analysis}
\cite{jimenez_soft_2012-1,jimenez_softcardinality:_2013}.
%\cite{jimenez_soft_2012-1,jimenez_sergio_softcardinality:_2013,jimenez_softcardinality:_2013}.
Although, this method is based purely in string matching, the soft
cardinality has been shown to be a very strong baseline for semantic textual
comparison. Clearly, the word-to-word similarity $sim$ in Eq.~\ref{eq:soft_card}
could be replaced by other similarity function based on semantic networks
or any distributional representation making this method able to capture
more complex semantic relations among words. Similarly, Croce et al.
used soft cardinality representing text as a bag of dependencies (syntactic
soft cardinality \cite{croce_distributional_2012}) obtaining the
best results in the typed-similarity task \cite{croce_unitor-core_2013}. 

%%% Local Variables: 
%%% mode: latex
%%% TeX-master: "sts14-ntnu"
%%% End: 
