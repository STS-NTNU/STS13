\subsection{Soft cardinality measures}
\label{sec:softcard}

Soft cardinality resembles classical set cardinality as it is a method
for counting the number of elements in a set, but differs from it
in that similarities among elements are being considered for
the ``soft counting''. 
The soft cardinality of a set of words 
$A=\{a_{1},a_{2},..,a_{|A|}\}$  (a sentence) is defined by:

\begin{equation}
|A|_{sim}=\sum_{i=1}^{|A|}\frac{w_{a_{i}}}{\sum_{j=1}^{|A|}sim(a_{i},a_{j})^{p}}\label{eq:soft_card}
\end{equation}

\noindent
Where $p$ is a parameter that controls the cardinality's softness ($p$'s default value is 1)
and $w_{a_{i}}$ are weights for each word,
obtained through inverse document frequency (\emph{idf}) weighting.
$sim(a_{i},a_{j})$ is a similarity function that compares two words $a_{i}$ and $a_{j}$
using the symmetrized Tversky's index \cite{tversky_features_1977,jimenez_softcardinality_core:_2013}
representing them as sets of 3-grams of characters. 
That is, $a_{i}=\{a_{i,1},a_{i,2},...,a_{i,|a_{i}|}\}$
where $a_{i,n}$ is the $n^{\rm th}$ character trigram in the word $a_{i}$ in $A$. 
Thus, the proposed word-to-word similarity is given by:

\begin{equation}
\hspace*{-.5em}sim(a_{i},a_{j}) {=} \frac{|c|}{\beta(\alpha|a_{min}|{+}(1{-}\alpha)|a_{max}|){+}|c|}\label{eq:symm_tversky}
\end{equation}
\vspace*{-2ex}\begin{equation*}
\begin{cases}
|c| &= |a_{i}\cap a_{j}|+bias_{sim}\\
|a_{min}| &= \min{|a_{i}\setminus a_{j}|,|a_{j}\setminus a_{i}|}\\
|a_{max}| &= \max{|a_{i}\setminus a_{j}|,|a_{j}\setminus a_{i}}
\end{cases}
\end{equation*}

The $sim$ function is equivalent to the Dice's coefficient
if the three parameters are given their default values, namely
$\alpha=0.5$, $\beta=1$ and $bias=0$.

The soft cardinalities of any pair of sentences $A$, $B$
and $A\cup B$ can be obtained using Eq.~\ref{eq:soft_card}. 
The soft cardinality of the intersection is approximated by 
$|A\cap B|_{sim}=|A|_{sim}+|B|_{sim}-|A\cup B|$.
These four basic soft cardinalities are algebraically recombined to produce
an extended set of 18 features as shown in Table~\ref{tab:features}. 

Although this method is based purely in string matching, the soft
cardinality has been shown to be a very strong baseline for semantic textual
comparison. The word-to-word similarity $sim$ in Eq.~\ref{eq:soft_card}
could be replaced by other similarity functions based on semantic networks
or any distributional representation making this method able to capture
more complex semantic relations among words. 

\begin{table}[t!]
\begin{centering}
\begin{tabular}{|c|c|c|c|}
\hline 
$\mathbf{STS_{sim}}$  & $\nicefrac{|A|-|A\cap B|}{|A|}$\tabularnewline
\hline 
{\small $|A|$} &  $\nicefrac{|A|-|A\cap B|}{|A\cup B|}$\tabularnewline
\hline 
{\small $|B|$} &  $\nicefrac{|B|}{|A\cup B|}$\tabularnewline
\hline 
{\small $|A\cap B|$} & $\nicefrac{|B|-|A\cap B|}{|B|}$\tabularnewline
\hline 
{\small $|A\cup B|$} &  $\nicefrac{|B|-|A\cap B|}{|A\cup B|}$\tabularnewline
\hline 
{\small $|A|-|A\cap B|$} & $\nicefrac{|A\cap B|}{|A|}$\tabularnewline
\hline 
{\small $|B|-|A\cap B|$}  & $\nicefrac{|A\cap B|}{|B|}$\tabularnewline
\hline 
{\small $|A\cap B|-|A\cap B|$} & $\nicefrac{|A\cap B|}{|A\cup B|}$\tabularnewline
\hline 
$\nicefrac{|A|}{|A\cup B|}$ & $\nicefrac{|A\cup B|-|A\cap B|}{|A\cap B|}$\tabularnewline
\hline
\end{tabular}
\\[1ex]
{\hfill\em\footnotesize NB: in this table only, $|*|$ is short for $|*|_{sim}$\hfill}
\end{centering}
\caption{Soft cardinality features\label{tab:features}}
\end{table}

%%% Local Variables: 
%%% mode: latex
%%% TeX-master: "sts14-ntnu"
%%% End: 
