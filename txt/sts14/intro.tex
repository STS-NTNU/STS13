\section{Introduction}
\label{intro}

\blfootnote{
     % final paper: en-us version (to licence, a license)
    
    \hspace{-0.65cm}  % space normally used by the marker
    This work is licenced under a Creative Commons
    Attribution 4.0 International License.
    Page numbers and proceedings footer are added by
    the organizers.
    License details:
    \url{http://creativecommons.org/licenses/by/4.0/}
}

This paper describes the NTNU submission to the SemEval 2014 Semantic Textual Similarity (STS) shared task. It is based on  the baseline Takelab system from the 2012 shared task and improving it in three aspects. It adds two new categories of features and uses a bagging regression model to predict similarity scores. The shared task evaluation indicates that these improvements gives a improvement to the baseline system performance.

% something about Parthas features
Based on earlier results in this task we have added features based on soft cardinality and character n-grams. Soft cardinality similarity measures has shown themselves to be strong features in earlier submissions \cite{jimenez_softcardinality-core:_2013} and the NTNU system uses an ensemble of 18 such measures using different similarity functions.
We have also incorperated semantic similarity metrics based on character n-gram feature representations. These representations replace character n-gram features with clusters or vectors based on the n-gram collocational structure learned in an unsupervised manner from text data. Three different techniques are used to induce such representations: Brown clustering \cite{brown1992class}, log linear skip-gram representations \cite{mikolov2013efficient} and Latent Semantic Indexing (LSI) topic vectors \cite{deerwester1990indexing}. We trained a variety of such feature representations on subsets of Wikipedia and used the best performing ones in our new similarity measures, which is a simple cosine similarity between the document vectors derived from each sentence of the given pair.

Combined with the baseline features the features based on these new measures shows competitive performance in the final evaluation results of the shared task.

%%% Local Variables:  
%%% mode: latex 
%%% TeX-master: "sts14-ntnu" 
%%% End: 
