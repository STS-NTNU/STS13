\section{Introduction}
\label{intro}

%
% The following footnote without marker is needed for the camera-ready
% version of the paper.
% Comment out the instructions (first text) and uncomment the 8 lines
% under "final paper" for your variant of English.
% 
\blfootnote{
    %
    % for review submission
    %
    % \hspace{-0.65cm}  % space normally used by the marker
    % {\em Place for licence statement for the camera-ready version.}
    %
    % % final paper: en-uk version (to license, a licence)
    %
    \hspace{-0.65cm}  space normally used by the marker
    This work is licensed under a Creative Commons 
    Attribution 4.0 International Licence.
    Page numbers and proceedings footer are added by
    the organisers.
    Licence details:
    \url{http://creativecommons.org/licenses/by/4.0/}
    % 
    % % final paper: en-us version (to licence, a license)
    %
    % \hspace{-0.65cm}  % space normally used by the marker
    % This work is licenced under a Creative Commons 
    % Attribution 4.0 International License.
    % Page numbers and proceedings footer are added by
    % the organizers.
    % License details:
    % \url{http://creativecommons.org/licenses/by/4.0/}
}

The Semantic Textual Similarity (STS) shared task aims at providing a unified framework for evaluating textual semantic similarity, ranging from exact semantic equivalence to completely unrelated texts. This is represented by the prediction of a similarity score between two sentences, drawn from a particular category of text, which ranges from 0 (different topics) to 5 (exactly equivalent) through six grades of semantic similarity \cite{agirre-EtAl:2013:*SEM1}.
This paper describes the NTNU submission to the SemEval 2014 STS shared task (Task~10). 
The approach is based on the lexical and distributional features of the baseline TakeLab system 
from the 2012 shared task \cite{saric2012takelab} and improves on it in three ways,
by adding two new categories of features and by using a bagging regression model to predict similarity scores. 

The new feature categories added are based on soft cardinality and character n-grams,
described in Section~\ref{sec:methods}.
The parameters of the two categories are optimised over several corpora and the features
combined through support vector regression (Section~\ref{sec:optimisation})
to create the actual systems (Section~\ref{sec:systems}).
As Section~\ref{sec:results} shows, the new measures give 
the baseline system a substantial boost, leading to
very competitive results in the shared task evaluation.

%%% Local Variables:  
%%% mode: latex 
%%% TeX-master: "sts14-ntnu" 
%%% End: 
