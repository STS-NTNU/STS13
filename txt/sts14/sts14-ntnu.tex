%
% File twocolumn.tex
%
%%
%% Based on the style files for *SEM-2014, which were, in turn,
%% Based on the style files for COLING-2014, which were, in turn,
%% Based on the style files for ACL-2014, which were, in turn,
%% Based on the style files for ACL-2013, which were, in turn,
%% Based on the style files for ACL-2012, which were, in turn,
%% based on the style files for ACL-2011, which were, in turn,
%% based on the style files for ACL-2010, which were, in turn,
%% based on the style files for ACL-IJCNLP-2009, which were, in turn,
%% based on the style files for EACL-2009 and IJCNLP-2008...

%% Based on the style files for EACL 2006 by
%%e.agirre@ehu.es or Sergi.Balari@uab.es
%% and that of ACL 08 by Joakim Nivre and Noah Smith

\documentclass[11pt]{article}
\usepackage{semeval2014}
\usepackage{times}
\usepackage{url}
\usepackage{latexsym}

%\setlength\titlebox{5cm}

% You can expand the titlebox if you need extra space
% to show all the authors. Please do not make the titlebox
% smaller than 5cm (the original size); we will check this
% in the camera-ready version and ask you to change it back.

\newcommand{\wsname}{SemEval-2014}
\newcommand{\submissionpage}{\url{http://alt.qcri.org/semeval2014/index.php?id=cfp}}
\newcommand{\filename}{semeval2014}
\newcommand{\contact}{pnakov qf.org.qa}

\title{NTNU: Measuring semantic similarity with sublexical feature representations}

\author{Andr\'{e} Lynum \\
  \\
  \\
  \\
  \\
  \And
  Partha Pakray \\
  {\tt \{andrely,parthap,gamback\}@idi.ntnu.no} \\
  Norwegian University of Science and Technology \\
  Sem S{\ae}landsvei 7-9 \\
  Trondheim, Norway \\
  \And
  Bj\"{o}rn Gamb\"{a}ck \\
  \\
  \\
  \\
  \\
}

\date{}

% remember to put license back in

\begin{document}
\maketitle
\begin{abstract}
  This paper describes the system submitted by the NTNU team to the SemEval 2014 shared task 10 (Semantic Textual Similarity). It combines baseline semantic distance metrics from the 2013 submission with measures based on lexical soft cardinality and character n-gram feature representations. The final system is based on bagged support vector machine (SVM) regression on earlier shared task datasets.
\end{abstract}

\section{Introduction}
\label{intro}

\blfootnote{
     % final paper: en-us version (to licence, a license)
    
    \hspace{-0.65cm}  % space normally used by the marker
    This work is licenced under a Creative Commons
    Attribution 4.0 International License.
    Page numbers and proceedings footer are added by
    the organizers.
    License details:
    \url{http://creativecommons.org/licenses/by/4.0/}
}

This paper describes the NTNU submission to the SemEval 2014 Semantic Textual Similarity (STS) shared task. It is based on  the baseline Takelab system from the 2012 shared task and improving it in three aspects. It adds two new categories of features and uses a bagging regression model to predict similarity scores. The shared task evaluation indicates that these improvements gives a improvement to the baseline system performance.

% something about Parthas features
Based on earlier results in this task we have added features based on soft cardinality and character n-grams. Soft cardinality similarity measures has shown themselves to be strong features in earlier submissions \cite{jimenez_softcardinality-core:_2013} and the NTNU system uses an ensemble of 18 such measures using different similarity functions.
We have also incorperated semantic similarity metrics based on character n-gram feature representations. These representations replace character n-gram features with clusters or vectors based on the n-gram collocational structure learned in an unsupervised manner from text data. Three different techniques are used to induce such representations: Brown clustering \cite{brown1992class}, log linear skip-gram representations \cite{mikolov2013efficient} and Latent Semantic Indexing (LSI) topic vectors \cite{deerwester1990indexing}. We trained a variety of such feature representations on subsets of Wikipedia and used the best performing ones in our new similarity measures, which is a simple cosine similarity between the document vectors derived from each sentence of the given pair.

%%% Local Variables:  
%%% mode: latex 
%%% TeX-master: "sts14-ntnu" 
%%% End: 



\section{Soft cardinality measures}
\label{sec:softcard}



%%% Local Variables: 
%%% mode: latex
%%% TeX-master: "sts14-ntnu"
%%% End: 


\section{Sublexical feature representations}
\label{subrep-features}

We have created a set of similarity measures based on induced representations of character n-grams. The measures are based on similarity between document vectors, here the centroid of the individual term vector representations, which are trained on character n-grams rather than full words. The vector representation are induced in an unsupervised manner from large unannotated corpora using word clustering, topic learning and word representation learning methods.

In order to learn the representation vectors, the text was preprocessed by removing punctuation and extra whitespace, replacing numbers with their single digit word (`one', `two', etc.), and lowercasing all text. Character n-grams including whitespace were then generated from this text. We then trained Brown Clusters \cite{brown1992class}, Latent Semantic Indexing (LSI) topics \cite{deerwester1990indexing}, and log linear skip-gram models \cite{mikolov2013efficient} on the resulting character n-grams. The representation vectors were trained on subsets of Wikipedia consisting of the first 12 million words (or $10^8$ characters) referred to as {\it Wiki8} and 125 million words ($10^9$ characters) referred to as {\it Wiki9}. In relation to the overall system pipeline this can be considered an unsupervised training step.
The representation vectors were trained varying certain parameters: n-gram size, cluster size, and term frequency cut-offs for all models.
For log linear skip-gram models our intuition is that a larger skip-gram context is needed than the 5 or 10 wide skip-grams 
used to train word-based representations due to the smaller term vocabulary and dependency between adjacent n-grams.
 For these measures, we trained models using skip-gram widths of 25 or 50 terms. 
Term frequency cut-offs were set to limit the model size, but also potentially serve as a regularization on the resulting measure.

The Brown clusters were trained using Percy Liangs software implementation \cite{liang2005semi}, while the LSI topic vectors and log linear skip-gram representations were trained using the Gensim topic modelling framework \cite{gensim_lrec}. 
In addition, TF-IDF (Term-Frequency Inverse Document Frequency) weighting was used when training LSI topic models. 
We used a cosine distance measure between document vectors consisting of the centroid of the term representation vectors. 
For Brown clusters, the normalized term frequency vectors were used with the cluster ids instead of the terms themselves. 
For LSI topic representations, the TF-IDF weighted topic mixture for each term was used as the term representation. 
For the log linear skip-gram model, the word representations were extracted from the model weight matrix.

%%% Local Variables:  
%%% mode: latex 
%%% TeX-master: "sts14-ntnu" 
%%% End: 


\section{Similarity score regression}
\label{sec:regression}

The final sentence pair similarity score is predicted by a Support Vector Machine (SVM) regression model using a Radial Basis (RBF) kernel \cite{VapnikEA:97}. The model is trained on the combined test data for the 2013 STS shared task and the trial and test data of the SemEval 2012 STS shared task. The combined dataset consists of ca. 7500 sentence pairs from nine different sources. The regression model is trained as a bagged classifier. I.e. for each run 100 regression models are trained with 80\% of the training samples and features of the original training set drawn with replacement. The prediction of all these regression models are averaged into a final prediction. This bagged training procedure introduces additional regularization, that can  reduce the instability of prediction accuracy between different test data categories. The prediction pipeline was implemented with the Scikit-learn software framework \cite{scikit-learn}.

The SVM regression models were trained with default parameters provided in the implementation, which are Cost penalty (C) 1.0, margin ($\epsilon$) 0.1 and RBF precision ($\gamma$) $1/|feature count|$. We could not msanage to improve the performance over these defaults by cross validation parameter search unless the models were trained for specific dataset categories.

%%% Local Variables:  
%%% mode: latex
%%% TeX-master: "sts14-ntnu"
%%% End: 


\section{Submitted systems}
\label{sec:systems}

The three submitted systems consists of one using only the soft cardinality features (NTNU-run1), one using a baseline set of lexical measures and WordNet augmented similarity in addition to the new sublexical representation measures (NTNU-run2) and one which combines the measures of the two other systems (NTNU-run3).

{\bf NTNU-run1} uses only the features and scoring algorithm described in section \ref{sec:softcard}.

{\bf NTNU-run2} uses the following baseline features adapted from the Takelab 2012 system submission \cite{saric2012takelab}.

\begin{itemize}
\item Simple lexical features: Relative document length differences, number overlab, case overlap and stock symbol named entity recognition.
\item Lemma and word n-gram overlap of orders 1-3.
\item Frequency weighted lemma and word overlap.
\item WordNet augmented overlap.
\item Cosine similarity between the summed word representation vectors from each sentence using LSI models based on large corpora with or without frequency weighting.
\end{itemize}

In addition the systems uses the following new measures based on sublexical word representations generated as described in section \ref{subrep-features}.

\begin{itemize}
\item Loglinear skip-gram representations of character 3- and 4-grams of size 1000 and 2000 respectively. Trained on the Wiki8 corpus using a skip gram window of size 25 and 50 and frequency cutoff of 5 .
\item Brown clusters of character 4-grams with cluster size 1024 using a frequency cutoff of 20.
\item Brown clusters of character 3-, 4- and 5-grams with cluster sizes of respectively 1024, 2048 and 1024. Representations are trained on the Wiki9 corpus with successively increasing frequency cutoffs of 20, 320 and 1200.
\item LSI topic vectors based on character 4-grams of size 2000.   Trained on the Wiki8 corpus using a frequency cutoff of 5.
\item LSI topic vectors based on character 4-grams of size 1000. Trained on the Wiki9 corpus using a frequency cutoff of 80.
\end{itemize}

% from subrep

The specific measures used in the submitted systems were found by by training the regression model on the STS 2012 shared task data and evaluating on the STS 2013 test data. We used a stepwise foreward feature selection method by comparing mean (but unweighted) correlation on the four test categories in order to identify the subset of measures to include in the final system.

The system composes a feature set of similarity scores from these 20 baseline measures and the nine sublexical representation measures, and uses these to train a bagged SVM regressor as described in section \ref{sec:regression} in order to predict the final semantic similarity score for new sentence pairs.

{\bf NTNU-run3} combines the output from NTNU-run1 and NTNU-run2 by taking the mean of the two sets of predictions. As such this system represents a combination of the measures and methods introduced by NTNU-run1 and NTNU-run2.


%%% Local Variables: 
%%% mode: latex
%%% TeX-master: "sts14-ntnu"
%%% End: 



\section{Results}
\label{sec:results}

\begin{table*}
  \centering
  \begin{tabular}{|l|c|c|c|}
    \hline
    Category & NTNU-run1 & NTNU-run2 & NTNU-run3 \\
    \hline
    deft-forum & 0.4369 & 0.5084 & 0.5305 \\
    deft-news & 0.7138 & 0.7656 & 0.7813 \\
    headlines & 0.7219 & 0.7525 & 0.7837 \\
    images & 0.8000 & 0.8129 & 0.8343 \\
    OnWN & 0.8348 & 0.7767 & 0.8502 \\
    tweet-news & 0.4109 & 0.7921 & 0.6755	\\
    \hline
    weighted mean & 0.6631 & 0.7491 & 0.7549 \\
    \hline
  \end{tabular}
  \caption{Final evaluation results for the submitted systems.}
  \label{tab:results}
\end{table*}

The final evaluation results for the three submitted systems are shown in table \ref{tab:results}. The systems using sublexical representation based measures show competitive performance, ranking third and fourth among the submitted systems with a weighted mean correlation of ca. $0.75$. The system based solely on soft cardinality features display more modest performance ranking at 21st. place with ca. 0.66 correlation. Out of the results one can note that for most categories the sublexical representation measures shows strong performance in NTNU-run2, with a significantly stronger result for the combined system NTNU-run3. This indicates that while the soft cardinality features are weaker predictors overall, they are complimentary to the sublexical ands lexical features of NTNU-run2. It is also indicative that this is not the case for the tweet-news category, where the text is more ``free form'' and less normative and one would expect sublexical approaches to have stronger performance.

%%% Local Variables: 
%%% mode: latex
%%% TeX-master: "sts14-ntnu"
%%% End: 


\bibliographystyle{acl}
\bibliography{sts14-ntnu}


\end{document}
