%
% File twocolumn.tex
%
%%
%% Based on the style files for *SEM-2014, which were, in turn,
%% Based on the style files for COLING-2014, which were, in turn,
%% Based on the style files for ACL-2014, which were, in turn,
%% Based on the style files for ACL-2013, which were, in turn,
%% Based on the style files for ACL-2012, which were, in turn,
%% based on the style files for ACL-2011, which were, in turn,
%% based on the style files for ACL-2010, which were, in turn,
%% based on the style files for ACL-IJCNLP-2009, which were, in turn,
%% based on the style files for EACL-2009 and IJCNLP-2008...

%% Based on the style files for EACL 2006 by
%%e.agirre@ehu.es or Sergi.Balari@uab.es
%% and that of ACL 08 by Joakim Nivre and Noah Smith

\documentclass[11pt]{article}
\usepackage{semeval2014}
\usepackage{times}
\usepackage{url}
\usepackage{latexsym}

%\setlength\titlebox{5cm}

% You can expand the titlebox if you need extra space
% to show all the authors. Please do not make the titlebox
% smaller than 5cm (the original size); we will check this
% in the camera-ready version and ask you to change it back.

\newcommand{\wsname}{SemEval-2014}
\newcommand{\submissionpage}{\url{http://alt.qcri.org/semeval2014/index.php?id=cfp}}
\newcommand{\filename}{semeval2014}
\newcommand{\contact}{pnakov qf.org.qa}

\title{NTNU: Measuring semantic similarity with sublexical feature representations}

\author{Andr\'{e} Lynum \\
  \\
  \\
  \\
  \\
  \And
  Partha Pakray \\
  {\tt \{andrely,parthap,gamback\}@idi.ntnu.no} \\
  Norwegian University of Science and Technology \\
  Sem S{\ae}landsvei 7-9 \\
  Trondheim, Norway \\
  \And
  Bj\"{o}rn Gamb\"{a}ck \\
  \\
  \\
  \\
  \\
}

\date{}

% remember to put license back in

\begin{document}
\maketitle
\begin{abstract}
  This paper describes the system submitted by the NTNU team to the SemEval 2014 shared task 10 (Semantic Textual Similarity). It combines baseline semantic distance metrics from the 2013 submission with measures based on lexical soft cardinality and character n-gram feature representations. The final system is based on bagged support vector machine (SVM) regression on earlier shared task datasets.
\end{abstract}

\section{Introduction}
\label{intro}

\blfootnote{
     % final paper: en-us version (to licence, a license)
    
    \hspace{-0.65cm}  % space normally used by the marker
    This work is licenced under a Creative Commons
    Attribution 4.0 International License.
    Page numbers and proceedings footer are added by
    the organizers.
    License details:
    \url{http://creativecommons.org/licenses/by/4.0/}
}

This paper describes the NTNU submission to the SemEval 2014 Semantic Textual Similarity (STS) shared task. It is based on  the baseline Takelab system from the 2012 shared task and improving it in three aspects. It adds two new categories of features and uses a bagging regression model to predict similarity scores. The shared task evaluation indicates that these improvements gives a improvement to the baseline system performance.

% something about Parthas features
Based on earlier results in this task we have added features based on soft cardinality and character n-grams. Soft cardinality similarity measures has shown themselves to be strong features in earlier submissions \cite{jimenez_softcardinality-core:_2013} and the NTNU system uses an ensemble of 18 such measures using different similarity functions.
We have also incorperated semantic similarity metrics based on character n-gram feature representations. These representations replace character n-gram features with clusters or vectors based on the n-gram collocational structure learned in an unsupervised manner from text data. Three different techniques are used to induce such representations: Brown clustering \cite{brown1992class}, log linear skip-gram representations \cite{mikolov2013efficient} and Latent Semantic Indexing (LSI) topic vectors \cite{deerwester1990indexing}. We trained a variety of such feature representations on subsets of Wikipedia and used the best performing ones in our new similarity measures, which is a simple cosine similarity between the document vectors derived from each sentence of the given pair.

Combined with the baseline features the features based on these new measures shows competitive performance in the final evaluation results of the shared task.

%%% Local Variables:  
%%% mode: latex 
%%% TeX-master: "sts14-ntnu" 
%%% End: 



\section{Soft cardinality measures}
\label{sec:softcard}

A set of features, based only on surface text information,
were extracted using soft cardinality \cite{chavez_text_2010}, which
have been used successfully for the STS task in previous SemEval editions
\cite{jimenez_soft_2012,jimenez_softcardinality_core:_2013}. Soft
cardinality resembles classical set cardinality as it is a method
for counting the number of elements in a set but differs from the
later in that similarities among elements are being considered for
that ``soft counting''. The expression for the soft cardinality of a set of words $A=\{a_{1},a_{2},..,a_{|A|}\}$
(i.e. a text sentence) is as follows:

\begin{equation}
|A|_{sim}=\sum_{i=1}^{|A|}\frac{w_{a_{i}}}{\sum_{j=1}^{|A|}sim(a_{i},a_{j})^{p}}\label{eq:soft_card}
\end{equation}


Where $sim(a_{i},a_{j})$ is a similarity function between two words,
$w_{a_{i}}$ are weights for each word and $p$ is the softness control
parameter of the soft cardinality ($p$'s default value is 1). The
similarity function $sim$ compares two words $a_{i}$ and $a_{j}$
using the symmetrized Tversky's index \cite{tversky_features_1977,jimenez_softcardinality-core:_2013}
representing them as sets of 3-grams of characters. That is, $a_{i}=\{a_{i,1},a_{i,2},...,a_{i,|a_{i}|}\}$
where $a_{i,n}$ is the $n$-th character trigram in the word $a_{i}$
in $A$. Thus, the proposed word-to-word similarity function is given
by:

\begin{center}
\begin{equation}
sim(a_{i},a_{j})=\frac{|c|}{\beta(\alpha|a_{min}|+(1-\alpha)|a_{max}|)+|c|}\label{eq:symm_tversky}
\end{equation}

\par\end{center}

\begin{center}
$|c|=|a_{i}\cap a_{j}|+bias_{sim}$,
\par\end{center}

\begin{center}
$|a_{min}|=\min[|a_{i}\setminus a_{j}|,|a_{j}\setminus a_{i}|]$,
\par\end{center}

\begin{center}
$|a_{max}|=\max[|a_{i}\setminus a_{j}|,|a_{j}\setminus a_{i}|]$.
\par\end{center}

The parameters of this model are: $\alpha$, $\beta$ and $bias$,
whose default values $\alpha=0.5$, $\beta=1$ and $bias=0$ makes
the $sim$ function equivalent to the Dice's coefficient. The word
weights $w_{a_{i}}$ for each word were obtained using the well-known
\emph{idf} weighting schema.

Now, the soft cardinalities of any pair of text sentences $A$, $B$
and $A\cup B$ can be obtained using eq.\ref{eq:soft_card}. The soft
cardinality of the intersection is approximated by $|A\cap B|_{sim}=|A|_{sim}+|B|_{sim}-|A\cup B|$.
These 4 basic soft cardinalities are algebraically recombined to produce
an extended set of 18 features as shown in Table \ref{tab:features}. 

The feature \#1, $\mathbf{STS_{sim}}$ were used to optimize the 4
parameters $\alpha$, $\beta$, $bias$ and $p$ in the following
way. First, we build a text similarity function reusing eq.\ref{eq:symm_tversky}
for comparing two sets of words (instead of two sets of character
3-grams) and replacing the classic cardinality $|*|$ by the soft
cardinality $|*|_{sim}$ from eq.\ref{eq:soft_card}. This text similarity
function adds 3 parameters ($\alpha'$, $\beta'$, and $bias'$) to
the initial set of 4 parameters $\alpha$, $\beta$, $bias$ and $p$.
Second, these 7 parameters where set to their default values and the
scores obtained from this function for each pair of texts were compared
with their gold standard in the training data using Pearson's correlation.
Finally, iteratively, the parameters search space was explored using
hill-climbing until optimal Pearson's correlation is reached. As no
training data was explicitly provided for the STS evaluation campaign
this year, we used different training sets from past campaigns for
the new test sets. The criteria for the assignment of training-test
sets pairs was by their closeness of average character length. The
selected training-test sets pairs are shown in Table \ref{tab:training-test-sets}
and the optimal training parameters are shown in Table \ref{tab:Optimal-parameters}.

Before using extracting the proposed features, all texts were preprocessed
by: \emph{i)} tokenization and stop-word removal (provided by NLTK%
\footnote{http://www.nltk.org/%
}), \emph{ii)} conversion to lowercase characters, \emph{iii)} punctuation
and special character removal (e.g. ``.'', ``;'', ``\$'', ``\&''),
and \emph{iv) }stemming using Porter's algorithm. Next, this preprocessed
output is used to obtain the \emph{idf} weights and the features described
above.

\begin{table}
\begin{centering}
\begin{tabular}{|c|c|c|c|}
\hline 
{\footnotesize \#1} & $\mathbf{STS_{sim}}$ & {\footnotesize \#10} & $\nicefrac{|A|-|A\cap B|}{|A|}$\tabularnewline
\hline 
{\footnotesize \#2} & {\small $|A|$} & {\footnotesize \#11} & $\nicefrac{|A|-|A\cap B|}{|A\cup B|}$\tabularnewline
\hline 
{\footnotesize \#3} & {\small $|B|$} & {\footnotesize \#12} & $\nicefrac{|B|}{|A\cup B|}$\tabularnewline
\hline 
{\footnotesize \#4} & {\small $|A\cap B|$} & {\footnotesize \#13} & $\nicefrac{|B|-|A\cap B|}{|B|}$\tabularnewline
\hline 
{\footnotesize \#5} & {\small $|A\cup B|$} & {\footnotesize \#14} & $\nicefrac{|B|-|A\cap B|}{|A\cup B|}$\tabularnewline
\hline 
{\footnotesize \#6} & {\small $|A|-|A\cap B|$} & {\footnotesize \#15} & $\nicefrac{|A\cap B|}{|A|}$\tabularnewline
\hline 
{\footnotesize \#7} & {\small $|B|-|A\cap B|$} & {\footnotesize \#16} & $\nicefrac{|A\cap B|}{|B|}$\tabularnewline
\hline 
{\footnotesize \#8} & {\small $|A\cap B|-|A\cap B|$} & {\footnotesize \#17} & $\nicefrac{|A\cap B|}{|A\cup B|}$\tabularnewline
\hline 
{\footnotesize \#9} & $\nicefrac{|A|}{|A\cup B|}$ & {\footnotesize \#18} & $\nicefrac{|A\cup B|-|A\cap B|}{|A\cap B|}$\tabularnewline
\hline 
\end{tabular}
\par\end{centering}

\begin{centering}
Note: for short, only in this table $|*|$ stands for $|*|_{sim}$
\par\end{centering}

\caption{Soft cardinality features\label{tab:features}}
\end{table}


\begin{table}
\begin{centering}
\begin{tabular}{|c|c|}
\hline 
{\scriptsize 2014 Test set} & {\scriptsize Training set}\tabularnewline
\hline 
\hline 
{\scriptsize OnWN} & {\scriptsize OnWN 2012 and 2013 test}\tabularnewline
\hline 
{\scriptsize headlines} & {\scriptsize headlines 2013 test}\tabularnewline
\hline 
{\scriptsize images} & {\scriptsize MSRvid 2012 train + test}\tabularnewline
\hline 
{\scriptsize deft-news} & {\scriptsize MSRvid 2012 train + test}\tabularnewline
\hline 
\multirow{2}{*}{{\scriptsize deft-forum}} & {\scriptsize MSRvid 2012 train and test +}\tabularnewline
 & {\scriptsize OnWN 2012 and 2013 test}\tabularnewline
\hline 
\multirow{2}{*}{{\scriptsize tweet-news}} & {\scriptsize SMTeuroparl 2012 test +}\tabularnewline
 & {\scriptsize SMTnews 2012 test }\tabularnewline
\hline 
\end{tabular}
\par\end{centering}

\centering{}\caption{Used training-test sets pairs\label{tab:training-test-sets}}
\end{table}


\begin{table}
\centering{}%
\begin{tabular}{|c|c|c|c|c|c|c|c|}
\hline 
{\footnotesize dataset} & {\footnotesize $\alpha$} & {\footnotesize $\beta$} & {\footnotesize $bias$} & {\footnotesize $p$} & {\footnotesize $\alpha'$} & {\footnotesize $\beta$} & {\footnotesize $bias'$}\tabularnewline
\hline 
\hline 
{\scriptsize OnWN} & {\scriptsize 0.53} & {\scriptsize -0.53} & {\scriptsize 1.01} & {\scriptsize 1.00} & {\scriptsize -4.89} & {\scriptsize 0.52} & {\scriptsize 0.46}\tabularnewline
\hline 
{\scriptsize headlines} & {\scriptsize 0.36} & {\scriptsize -0.29} & {\scriptsize 4.17} & {\scriptsize 0.85} & {\scriptsize -4.50} & {\scriptsize 0.43} & {\scriptsize 0.19}\tabularnewline
\hline 
{\scriptsize images} & {\scriptsize 1.12} & {\scriptsize -1.11} & {\scriptsize 0.93} & {\scriptsize 0.64} & {\scriptsize -0.98} & {\scriptsize 0.50} & {\scriptsize 0.11}\tabularnewline
\hline 
{\scriptsize deft-news} & {\scriptsize 3.36} & {\scriptsize -0.64} & {\scriptsize 1.37} & {\scriptsize 0.44} & {\scriptsize 2.36} & {\scriptsize 0.72} & {\scriptsize 0.02}\tabularnewline
\hline 
{\scriptsize deft-forum} & {\scriptsize 1.01} & {\scriptsize -1.01} & {\scriptsize 0.24} & {\scriptsize 0.93} & {\scriptsize -2.71} & {\scriptsize 0.42} & {\scriptsize 1.63}\tabularnewline
\hline 
{\scriptsize tweet-news} & {\scriptsize 0.13} & {\scriptsize 0.14} & {\scriptsize 2.80} & {\scriptsize 0.01} & {\scriptsize 2.66} & {\scriptsize 1.74} & {\scriptsize 0.45}\tabularnewline
\hline 
\end{tabular}\caption{Optimal parameters used for each dataset\label{tab:Optimal-parameters}}
\end{table}
This method for obtaining features from pairs of texts was also used
successfully in other SemEval tasks such as \emph{cross-lingual textual
entailment }and \emph{student response analysis} \cite{jimenez_soft_2012-1,jimenez_sergio_softcardinality:_2013,jimenez_softcardinality:_2013}.
Although, this method is based purely in string matching, the soft
cardinality have shown to be a very strong baseline for semantic textual
comparison. Clearly, the word-to-word similarity $sim$ in eq.\ref{eq:soft_card}
could be replaced by other similarity function based on semantic networks
or any distributional representation making this method able to capture
more complex semantic relations among words. Similarly, Croce et al.
used soft cardinality representing text as a bag of dependencies (syntactic
soft cardinality \cite{croce_distributional_2012}) obtaining the
best results in the typed-similarity task \cite{croce_unitor-core_2013}. 

%%% Local Variables: 
%%% mode: latex
%%% TeX-master: "sts14-ntnu"
%%% End: 


\section{Sublexical feature representations}
\label{subrep-features}

We have created a set of similarity measures based on induced representations of character n-grams. The measures are based on similarity between document vectors, here the centroid of the individual term vector representations, which are trained on character n-grams rather than full words. The vector representation are induced in an unsupervised manner from large unannotated corpora using word clustering, topic learning and word representation learning methods.

In order to learn the representation vectors, the text was preprocessed by removing punctuation and extra whitespace, replacing numbers with their single digit word (`one', `two', etc.), and lowercasing all text. Character n-grams including whitespace were then generated from this text. We then trained Brown Clusters \cite{brown1992class}, Latent Semantic Indexing (LSI) topics \cite{deerwester1990indexing}, and log linear skip-gram models \cite{mikolov2013efficient} on the resulting character n-grams. The representation vectors were trained on subsets of Wikipedia consisting of the first 12 million words (or $10^8$ characters) referred to as {\it Wiki8} and 125 million words ($10^9$ characters) referred to as {\it Wiki9}. In relation to the overall system pipeline this can be considered an unsupervised training step.
The representation vectors were trained varying certain parameters: n-gram size, cluster size, and term frequency cut-offs for all models.
For log linear skip-gram models our intuition is that a larger skip-gram context is needed than the 5 or 10 wide skip-grams 
used to train word-based representations due to the smaller term vocabulary and dependency between adjacent n-grams.
 For these measures, we trained models using skip-gram widths of 25 or 50 terms. 
Term frequency cut-offs were set to limit the model size, but also potentially serve as a regularization on the resulting measure.

The Brown clusters were trained using Percy Liangs software implementation \cite{liang2005semi}, while the LSI topic vectors and log linear skip-gram representations were trained using the Gensim topic modelling framework \cite{gensim_lrec}. 
In addition, TF-IDF (Term-Frequency Inverse Document Frequency) weighting was used when training LSI topic models. 
We used a cosine distance measure between document vectors consisting of the centroid of the term representation vectors. 
For Brown clusters, the normalized term frequency vectors were used with the cluster ids instead of the terms themselves. 
For LSI topic representations, the TF-IDF weighted topic mixture for each term was used as the term representation. 
For the log linear skip-gram model, the word representations were extracted from the model weight matrix.

%%% Local Variables:  
%%% mode: latex 
%%% TeX-master: "sts14-ntnu" 
%%% End: 


\section{Similarity score regression}
\label{sec:regression}

The sentence pair similarity score is predicted by a Support Vector Machine (SVM) regression model using Radial Basis (RBF) kernel function. The model is trained on the combined test data for the 2013 STS shared task and the trial and test data of the SemEval 2012 STS shared task. The combined dataset consists of ca. 7500 sentence pairs from nine different sources. The regression model is trained as a bagged classifier. I.e. for each run 100 such regression models was built with a set of training samples and features with size to 80\% of the original features and samples drawn with replacement. The prediction of the regression models are averaged into a final prediction. This introduces additional regularization in an attempt to reduce the instability of results between different test sets.

The SVM regression models were trained with default parameters provided in the implementation, which are Cost penalty (C) 1.0, margin ($\epsilon$) 0.1 and RBF precision ($\gamma$) $1/|feature count|$. It was not possible to improve the performance over these defaults by cross validation parameter search unless the dataset was split up along the different data sources.



\section{Submitted systems}
\label{sec:systems}

The three submitted systems consists of one using negation measures alone (NTNU-run1), one using a baseline set of lexical measures and WordNet augmented similarity in addition to the new sublexical representation measures (NTNU-run2) and one which combines the measures of the two other systems (NTNU-run3).

NTNU-run2 uses the following baseline features adapted from the Takelab 2012 system submission \cite{saric2012takelab}.

\begin{itemize}
\item Simple lexical features: Relative document length differences, number overlab, case overlap and stock symbol named entity recognition.
\item Lemma and word n-gram overlap of orders 1-3.
\item Frequency weighted lemma and word overlap.
\item WordNet augmented overlap.
\item Cosine similarity between the summed word representation vectors from each sentence using LSI models based on large corpora with or without frequency weighhting.
\end{itemize}

In addition the systems uses the following new measures based on sublexical word representations generated as described in \ref{subrep-measures}.

\begin{itemize}
\item Loglinear skip-gram representations of character 3- and 4-grams of size 1000 and 2000 respectively. Trained on the Wiki8 corpus using a skip gram window of size 25 and 50 and frequency cutoff of 5 .
\item Brown clusters of character 4-grams with cluster size 1024 using a frequency cutoff of 20.
\item Brown clusters of character 3-, 4- and 5-grams with cluster sizes of respectively 1024, 2048 and 1024. Representations are trained on the Wiki9 corpus with successively increasing frequency cutoffs of 20, 320 and 1200.
\item LSI topic vectors based on character 4-grams of size 2000.   Trained on the Wiki8 corpus using a frequency cutodff of 5.
\item LSI topic vectors based on character 4-grams of size 1000. Trained on the Wiki9 corpus using a frequency cutodff of 80.
\end{itemize}

The system composes a feature set of scores from these 20 baseline measures and nine sublexical representation measures and trains a bagged SVM regressor as described in \ref{sec:regression} in order to predict the semantic similarity score for new sentence pairs.

NTNU-run3 is identical to NTNU-run2, but incorporates the measures from NTNU-run1 as features. As such this system represents a combination of the new measures and methods introduced by NTNU-run1 and NTNU-run2.

% from subrep

The specific measures used in the submitted systems were found by by training the regression model on the STS 2012 shared task data and evaluating on the STS 2013 test data. We used a stepwise forword selection method by comparing mean (but unweighted) correlation on the four test categories in order to identify the subset of measures to include in the final models.


%%% Local Variables: 
%%% mode: latex
%%% TeX-master: "sts14-ntnu"
%%% End: 



\section{Results}
\label{sec:results}

\begin{table*}
  \centering
  \begin{tabular}{|l|c|c|c|}
    \hline
    Category & NTNU-run1 & NTNU-run2 & NTNU-run3 \\
    \hline
    deft-forum & 0.4369 & 0.5084 & 0.5305 \\
    deft-news & 0.7138 & 0.7656 & 0.7813 \\
    headlines & 0.7219 & 0.7525 & 0.7837 \\
    images & 0.8000 & 0.8129 & 0.8343 \\
    OnWN & 0.8348 & 0.7767 & 0.8502 \\
    tweet-news & 0.4109 & 0.7921 & 0.6755	\\
    \hline
    weighted mean & 0.6631 & 0.7491 & 0.7549 \\
    \hline
  \end{tabular}
  \caption{Final evaluation results for the submitted systems.}
  \label{tab:results}
\end{table*}

The final evaluation results for the three submitted systems are shown in table \ref{tab:results}. The systems using sublexical representation based measures show competitive performance, ranking third and fourth among the submitted systems with a weighted mean correlation of ca. $0.75$. The system based solely on soft cardinality features display more modest performance ranking at 21st. place with ca. 0.66 correlation. Out of the results one can note that for most categories the sublexical representation measures shows strong performance in NTNU-run2, with a significantly stronger result for the combined system NTNU-run3. This indicates that while the soft cardinality features are weaker predictors overall, they are complimentary to the sublexical ands lexical features of NTNU-run2. It is also indicative that this is not the case for the tweet-news category, where the text is more ``free form'' and less normative and one would expect sublexical approaches to have stronger performance.

%%% Local Variables: 
%%% mode: latex
%%% TeX-master: "sts14-ntnu"
%%% End: 


\bibliographystyle{acl}
\bibliography{sts14-ntnu}


\end{document}
