\documentclass[11pt,letterpaper]{article}
\usepackage{naaclhlt2013}
\usepackage{times}
\usepackage{latexsym}
\usepackage{url}
\usepackage{todonotes}
% To hide all todo notes, simply do
%\usepackage[disable]{todonotes}
\usepackage{booktabs}
\usepackage{mdframed}
\usepackage{balance}

\setlength\titlebox{6.5cm}    % Expanding the titlebox

% Citations
\RequirePackage{natbib}

\newcommand{\rank}[1]{\footnotesize{#1}}
\newcommand{\feat}[1]{\texttt{#1}}

\hyphenation{Sem-Eval uni-gram uni-grams bi-gram bi-grams tri-gram tri-grams}

\title{NTNU-CORE: Combining strong features for semantic similarity}
% Paper Title: The title of your paper should be of the form "<ID>-<TASK>: <Paper Title>" where <ID> is the team ID used when registering with *SEM ST 2013, <TASK> is either CORE or TYPED, or CORE_TYPED (assuming you participated in both evaluations, and <Paper Title> is the title of your paper.

\author{Erwin Marsi, Hans Moen, Lars Bungum, Gleb Sizov, Bj\"orn Gamb\"ack, Andr\'e Lynum\\
  Norwegian University of Science and Technology\\
  Department of Computer and Information and Science\\
  Sem S{\ae}lands vei 7-9 \\
  NO-7491 Trondheim, Norway \\
  {\tt \{emarsi,hansmoe,bungum,sizov,gamback,andrely\}@idi.ntnu.no}}

\date{}


 %------------begin Float Adjustment
%two column float page must be 90% full
\renewcommand\dblfloatpagefraction{.90}
%two column top float can cover up to 80% of page
\renewcommand\dbltopfraction{.80}
%float page must be 90% full
\renewcommand\floatpagefraction{.90}
%top float can cover up to 80% of page
\renewcommand\topfraction{.80}
%bottom float can cover up to 80% of page
\renewcommand\bottomfraction{.80}
%at least 10% of a normal page must contain text
\renewcommand\textfraction{.1}
%separation between floats and text
\setlength\dbltextfloatsep{9pt plus 5pt minus 3pt }
%separation between two column floats and text
\setlength\textfloatsep{4pt plus 2pt minus 1.5pt}

\begin{document}
\maketitle
\begin{abstract}
The paper outlines the work carried out at NTNU as part of the 
*SEM'13 shared task on Semantic Textual Similarity,
using an approach which combines shallow textual, distributional 
and knowledge-based features by a support vector regression model. 
Feature sets include 
(1)~aggregated similarity based on named entity recognition with WordNet and
Levenshtein distance through the calculation of maximum weighted bipartite graphs; 
(2)~higher order word co-occurrence similarity using a novel method called 
``Multi-sense Random Indexing''; 
(3)~deeper semantic relations based on the RelEx semantic dependency relationship extraction system; 
(4)~graph edit-distance on dependency trees; 
(5)~reused features of the TakeLab and DKPro systems from the STS'12 shared task. 
The NTNU systems obtained 9th place overall (5th best team) and 1st place on the SMT data set.
\end{abstract}

\section{Introduction}

Intuitively, two texts are semantically similar if they roughly mean the same thing.
The task of formally establishing semantic textual similarity clearly is more complex. For a start,
it implies that we have a way to formally represent the intended meaning of all texts in all possible
contexts, and furthermore a way to measure the degree of equivalence between two such representations.
This goes far beyond the state-of-the-art for arbitrary sentence pairs, and several restrictions must be imposed.
The Semantic Textual Similarity (STS) task~\citep{AgirreEA:12,AgirreEA:13} limits the
comparison to isolated sentences only (rather than complete texts), and defines similarity
of a pair of sentences as the one assigned by human judges %using Amazon Mechanical Turk,
on a 0--5 scale (with 0 implying no relation and 5 complete semantic equivalence).
%
It is unclear, however, to what extent two judges would agree
on the level of similarity between sentences;
%$s_1$ and $s_2$: even if exposed to the sentences in the same order. 
\citet{AgirreEA:12} report figures on the agreement between the authors 
themselves of about 87--89\%.
%, but include no calculation of inter-annotator agreement
%in the form of, e.g., Krippendorff's {\em alpha\/} \citep{Krippendorff:04}.
%Notably, the (graded) semantic equivalence relation between two sentences is assumed to be 
%symmetrical, which implies that human judges always would assign the same degree of similarity
%to the sentence pair regardless of whether they read $s_1$ or $s_2$ first. 
%This is a quite strong assumption, and as far as we know, it has not been evaluated.
%\todo[inline]{EM: Personally, I don't care to much for above paragraph. It's criticism of the task rather than description of our system.} 

As in most language processing tasks, there are two overall ways to measure sentence similarity,
either by data-driven (distributional) methods or by knowledge-driven methods; 
in the STS'12 task the two approaches were used nearly equally much.
%\todo[inline]{EM: how does e.g. character n-gram fit in? Can you call that knowledge-based?}
Distributional models normally measure similarity in terms of word or word co-occurrence statistics, 
or through concept relations extracted from a corpus.
The basic strategy taken by NTNU in the STS'13 task was to use something of a 
``feature carpet bombing approach'' in the way of first automatically extracting as many 
potentially useful features as possible, using both knowledge and data-driven methods,
and then evaluating feature combinations on the data sets provided by the organisers of the shared task.

To this end, four different types of features were extracted.
The first (Section~\ref{gleb-feats}) aggregates similarity based on named entity recognition with
WordNet and Levenshtein distance by calculating maximum weighted bipartite graphs.
%
The second set of features (Section~\ref{hans-feats}) models higher order co-occurrence 
similarity relations using Random Indexing \citep{Kanerva2000}, 
both in the form of a (standard) sliding window approach and 
through a novel method called ``Multi-sense Random Indexing'' which aims to
separate the representation of different senses of a term from each other.
%
The third feature set (Section~\ref{lars-feats}) aims to capture deeper semantic relations using 
either the output of the RelEx semantic dependency relationship extraction system \citep{bioinflmu-254} 
or an in-house graph edit-distance matching system.
%
The final set (Section~\ref{reused}) is a straight-forward gathering of features from the systems
that fared best in STS'12: TakeLab from University of Zagreb~\citep{vsaric2012takelab}
and DKPro from Darmstadt's Ubiquitous Knowledge Processing Lab~\citep{bar2012ukp}.

As described in Section~\ref{system}, Support Vector Regression~\citep{VapnikEA:97} was used
for solving the multi-dimensional regression problem of combining all the extracted feature values.
Three different systems were created based on feature performance on 
the supplied development data. %supplied by the task organisers.
Section~\ref{results} discusses scores on the STS'12 and STS'13 test data.

\section{Compositional Word Matching}
\label{gleb-feats}

Compositional word matching similarity is based on a one-to-one alignment of words from the two sentences. The alignment is obtained by maximal weighted bipartite matching using several word similarity measures. In addition, we utilise named entity recognition and matching tools. In general, the approach is similar to the one described by \citet{Karnick2012}, with a different set of tools used. 
Our implementation relies on the ANNIE components in GATE~\citep{CunninghamEA:02} 
and will thus be referred to as \feat{GateWordMatch}.

The processing pipeline for \feat{GateWordMatch} is:  
(1)~tokenization by ANNIE English Tokeniser, 
(2)~part-of-speech tagging by ANNIE POS Tagger, 
(3)~lemmatization by GATE Morphological Analyser, 
(4)~stopword removal, 
(5)~named entity recognition based on lists by ANNIE Gazetteer, 
(6)~named entity recognition based on the JAPE grammar by the ANNIE NE Transducer, 
(7)~matching of named entities by ANNIE Ortho Matcher, 
(8)~computing WordNet and Levenstein similarity between words, 
(9)~calculation of a maximum weighted bipartite graph matching based on similarities from 7~and~8. 

Steps 1--4 are standard preprocessing routines. 
In step~5, named entities are recognised based on lists that contain locations, organisations, companies, newspapers, and person names, as well as date, time and currency units. 
In step~6, JAPE grammar rules are applied to recognise entities such as addresses, emails, dates, job titles, and person names based on basic syntactic and morphological features. 
Matching of named entities in step~7 is based on matching rules that check the type of named entity, and lists with aliases to identify entities as ``US'', ``United State'', and ``USA'' as the same entity. 

In step~8, similarity is computed for each pair of words from the two sentences. Words that are matched as entities in step~7 get a similarity value of $1.0$. 
For the rest of the entities and non-entity words we use \emph{LCH\/} \citep{Leacock1998} similarity, which is based on a shortest path between the corresponding senses in WordNet. Since word sense disambiguation is not used, we take the similarity between the nearest senses of two words.
%The \emph{LCH\/} measure is limited to nouns and verbs and does not support cross-POS (part of speech) similarity. 
For cases when the WordNet-based similarity cannot be obtained, a similarity based on the Levenshtein distance \citep{Levenshtein:66} is used instead. It is normalised by the length of the longest word in the pair. For the STS'13 test data set, named entity matching contributed to 4\% of all matched word pairs; LCH similarity to 61\%, and Levenshtein distance to 35\%.

In step~9, 
%a complete bipartite graph is constructed and the 
maximum weighted bipartite matching is computed using the Hungarian Algorithm \citep{Kuhn1955}. 
Nodes in the bipartite graph represent words from the sentences, 
and edges have weights that correspond to similarities between tokens obtained in step~8.
Weighted bipartite matching finds the one-to-one alignment that maximizes 
%the total similarity of the matching, which is 
the sum of similarities between aligned tokens. 
Total similarity normalised by the number of words in both sentences 
is used as the final sentence similarity measure.

\section{Hans feats}




%%% Local Variables: 
%%% mode: latex
%%% TeX-master: "sts13-ntnu-core"
%%% End: 


\section{Deeper Semantic Relations}
\label{lars-feats}

Two deep strategies were employed to accompany the shallow-processed feature sets.  
Two existing systems were used to provide the basis for these features, namely the 
RelEx system~\citep{bioinflmu-254} from the OpenCog initiative~\citep{Hart:2008:OSF:1566174.1566223}, 
and an in-house graph-edit distance system
developed for plagiarism detection~\citep{rokenes13}.

%\subsection{RelEx}

RelEx outputs syntactic trees, dependency graphs, and \emph{semantic frames\/} as this one
for the sentence \linebreak
``\emph{Indian air force to buy 126 Rafale fighter jets\/}'':
\vspace*{-2.5ex}
%\begin{quote}
{\footnotesize\texttt{
\begin{mdframed}[linewidth=2pt]
Commerce\_buy:Goods(buy,jet)\\
Entity:Entity(jet,jet)\\
Entity:Name(jet,Rafale)\\
Entity:Name(jet,fighter)\\
Possibilities:Event(hyp,buy)\\
Request:Addressee(air,you)\\
Request:Message(air,air)\\
Transitive\_action:Beneficiary(buy,jet)
\end{mdframed}}}
%\end{quote}
\vspace*{2.5ex}
\noindent
Three features were extracted from this:
first, if there was an exact match of the frame found in $s_1$ with $s_2$;
second, if there was a partial match until the first argument (\texttt{Commerce\_buy:Goods(buy});
and third if there was a match of the frame category (\texttt{Commerce\_buy:Goods}).

%\subsection{Graph-edit distance}

In STS'12, \citet{singh-bhattacharya-bhattacharyya:2012:STARSEM-SEMEVAL} 
matched Universal Networking Language (UNL) graphs against each other 
by counting matches of relations and universal words, while
\citet{bhagwani-satapathy-karnick:2012:STARSEM-SEMEVAL} calculated
WordNet-based word-level similarities and created a weighted bipartite graph 
(see Section~\ref{gleb-feats}).
%
The method employed here instead looked at the graph edit distance 
between dependency graphs obtained with the
Maltparser dependency parser \citep{Nivre06maltparser:a}.
Edit distance is the defined as the minimum of the sum of the costs of the edit operations (insertion, deletion and substitution of nodes) required to transform one graph into the other.
It is approximated with a fast but suboptimal algorithm based on bipartite graph matching through the Hungarian algorithm~\citep{RiesenBunke:2009}.
%Edit cost was calculated between two dependency graphs obtained with.  
%The nodes in the two graphs were then assigned to each other using the Hungarian algorithm.  
%The cost was determined by looking at the number of additions, deletions and substitutions needed to transform one node to the other (having the same dependencies).


\section{Reused Features}
\label{reused}

The TakeLab `\emph{simple\/}' system \citep{vsaric2012takelab} 
obtained 3rd place in overall Pearson correlation 
and 1st for normalized Pearson in STS'12.
The source code\footnote{\url{http://takelab.fer.hr/sts/}} was used to generate 
all its features, that is,
%which comprises 
\emph{n-gram overlap}, 
\emph{WordNet-augmented word overlap}, 
\emph{vector space sentence similarity}, 
\emph{normalized difference}, 
\emph{shallow NE similarity}, 
\emph{numbers overlap}, and 
\emph{stock index} features.%
\footnote{We did not use content n-gram overlap or skip n-grams.} 
This required the full LSA vector space models, which were kindly provided by the TakeLab team. 
The word counts required for computing Information Content were obtained from Google Books Ngrams.%
\footnote{\url{http://storage.googleapis.com/books/ngrams/books/datasetsv2.html}, 
version 20120701, with 468,491,999,592 words}

%The TakeLab system for measuring semantic text similarity
%\citep{vsaric2012takelab} was one of the most succesful systems in the
%STS 2012 shared task. There were two variants called the \emph{simple}
%and \emph{syntax} system. The \emph{simple} system obtained 3rd place
%for overall Pearson and 1st for normalized Pearson. We used the source
%code\footnote{\url{http://takelab.fer.hr/sts/}} to generate the
%features for the STS 2012 data as well as the STS 2013 test data~\citep{AgirreEA:13}. 
%The later required LSA vector space models, which were kindly provided by
%the TakeLab team, and word counts, which were obtained from Google
%Books Ngrams (version 20120701, with 468,491,999,592 words).%
%\footnote{\url{http://storage.googleapis.com/books/ngrams/books/datasetsv2.html}}

%Since the TakeLab features are described by \citet{vsaric2012takelab},
%they will be reviewed only briefly here. The\emph{ n-gram overlap
%  features} measure overlap in unigrams, bigrams and trigrams of
%lower-cased words and lemmas, filtering stopwords and non-words (we
%did not use content n-gram overlap or skip n-grams). The\emph{
%  WordNet-augmented word overlap feature} measures unigram overlap
%where the similarity between word pairs is defined as the harmonic
%mean of their WordNet path length similarities \cite{}. The weighted
%word overlap features measure unigram overlap for lower-cased words
%and lemmas, where each word is weighted according to its Information
%Content (IC) calculated on the basis of unigram counts from Google Books
%Ngrams. The \emph{vector space sentence similarity features} capture
%distributional similarity between words of both sentences. Vector
%space models are derived from two corpora --- the New York Times
%Annotated Corpus (NYT) and Wikipedia (wiki) --- using Latent Semantic
%Analysis \citep{DeerwesterDumaisFurnas:1990}. A sentence vector is
%obtained by summing the vector of each word in the sentence.  A
%weighted variant uses IC to weight the vector of each word before
%summation. Vector space sentence similarity is then computed as the
%cosine similarity between the sentence vectors. The \emph{normalized
%difference} features measure the normalized differences in
%sentence length (in lower-cased stopword-filtered words) and in
%aggregated information content (sum of IC over all lower-cased
%words). The \emph{shallow NE similarity feature} expresses unigram
%overlap in named entities, treating each capitalized word longer than
%one character as a named entity (excluding the first word in the
%sentence). Finally, the \emph{numbers overlap} features and the
%\emph{stock index} features measure the overlap and count of numerals
%and stock index symbols respectively. 
% EM: provide more details if space allows

The DKPro system \citep{bar2012ukp} obtained first place in STS'12 with the second run. 
We used the source code\footnote{\url{http://code.google.com/p/dkpro-similarity-asl/}} 
to generate features for the STS'12 and STS'13 data. 
Of the string-similarity features, we reused the
\emph{Longest Common Substring}, 
\emph{Longest Common Subsequence} (with and without normalization), and
\emph{Greedy String Tiling\/} measures. 
From the character/word n-grams features, we used 
\emph{Character n-grams} ($n=2,3,4$), 
\emph{Word n-grams by Containment w/o Stopwords} ($n=1,2$),  
\emph{Word n-grams by Jaccard} ($n=1,3,4$), and  
\emph{Word n-grams by Jaccard w/o Stopwords} ($n=2,4$). 
Semantic similarity measures include 
\emph{WordNet Similarity\/} based on the Resnik measure (two variants) and 
\emph{Explicit Semantic Similarity\/} based on WordNet, Wikipedia or Wiktionary. 
This means that we reused all features from DKPro run~1 
%of \citet{bar2012ukp}, 
except for \emph{Distributional Thesaurus}.
% EM: The two variants of the Wordnet features are not clearly described in the DKPro paper. What does "variants: complete texts + difference only" mean? Anyway, looking at the feature relevance scores, the two variants give exactly the same score, both on the 2012 and 2013 data.


\section{Systems}
\label{system}

Our systems follow previous submissions to the STS task \citep[e.g.,][]{vsaric2012takelab,Banea2012}
in that feature values are extracted for each sentence pair and combined with a gold standard score 
in order to train a Support Vector Regressor on the resulting regression task. 
A postprocessing step guarantees that all scores are in the $[0,5]$ range and equal~5 if the two sentences are identical. 
%
SVR has been shown to be a powerful technique for predictive data analysis when the primary goal is to approximate 
a {\em function}, since the learning algorithm is applicable to continuous classes.
Hence support vector {\em regression\/} differs from support vector machine {\em classification\/}
where the goal rather is to take a {\em binary\/} decision.
%; deciding on which of two classes a given data point belongs to.
The key idea in SVR is to use a cost function for building the model which tries to ignore noise in
training data (i.e., data which is too close to the prediction), so that the produced model in essence 
only depends on a more robust subset of the extracted features. 
%EM: If we need a ref to support that VSM are good for NLP: joachims1998tex 

%The task organisers supplied training material
%mainly from STS'12 which contained data from the Microsoft Research Paraphrase and 
%Video Description corpora, statistical machine translation system output (Europarl and news), 
%and mappings between senses in OntoNotes and WordNet.
%In addition the organisers supplied sample data from the core test datasets, consisting of
%news headlines %mined from several news sources by European Media Monitor 
%(\texttt{HeadLine}),
%mappings of senses from WordNet and OntoNotes (\texttt{OnWN})
%as well as from WordNet and FrameNet (\texttt{FNWN}),
%and an evaluation of statistical machine translation (\texttt{SMT}) 
%giving the output of an MT system and a corresponding human reference translation.
%All the data files consist of lines with two sentences and their manually assigned similarity score.

%Participants could send a maximum of three system runs.
%Participants will also provide a confidence score indicating their confidence level for the result returned for each pair.
%The output of the systems performance is evaluated using the Pearson product-moment correlation coefficient
%between the system scores and the human scores (Rubenstein and Goodenough, 1965). 

Three systems were created using the supplied annotated data based on Microsoft Research Paraphrase and Video description corpora (MSRpar and MSvid), statistical machine translation system output (SMTeuroparl and SMTnews), and sense mappings between OntoNotes and WordNet (OnWN).
The first system (NTNU1) includes all TakeLab and DKPro features 
plus the \feat{GateWordMatch} feature with the SVR in its default setting.%
\footnote{RBF kernel, $\epsilon=0.1$, $C=\#samples$, $\gamma=\frac{1}{\#features}$}  
The training material consisted of all annotated data available, 
except for the SMT test set, where it was limited to SMT\-europarl and SMT\-news. 
The NTNU2 system is similar to NTNU1, except that the training material for OnWN and FNWN 
excluded MSRvid and that the SVR parameter $C$ was set to 200. 
NTNU3 is similar to NTNU1 except that {\em all\/} features available are included.



\section{Results}
\label{sec:results}

\begin{table*}
  \centering
  \begin{tabular}{|l|c|c|c|}
    \hline
    Category & NTNU-run1 & NTNU-run2 & NTNU-run3 \\
    \hline
    deft-forum & 0.4369 & 0.5084 & 0.5305 \\
    deft-news & 0.7138 & 0.7656 & 0.7813 \\
    headlines & 0.7219 & 0.7525 & 0.7837 \\
    images & 0.8000 & 0.8129 & 0.8343 \\
    OnWN & 0.8348 & 0.7767 & 0.8502 \\
    tweet-news & 0.4109 & 0.7921 & 0.6755	\\
    \hline
    weighted mean & 0.6631 & 0.7491 & 0.7549 \\
    \hline
  \end{tabular}
  \caption{Final evaluation results for the submitted systems.}
  \label{tab:results}
\end{table*}

The final evaluation results for the three submitted systems are shown in table \ref{tab:results}. The systems using sublexical representation based measures show competitive performance, ranking third and fourth among the submitted systems with a weighted mean correlation of ca. $0.75$. The system based solely on soft cardinality features display more modest performance ranking at 21st. place with ca. 0.66 correlation. Out of the results one can note that for most categories the sublexical representation measures shows strong performance in NTNU-run2, with a significantly stronger result for the combined system NTNU-run3. This indicates that while the soft cardinality features are weaker predictors overall, they are complimentary to the sublexical ands lexical features of NTNU-run2. It is also indicative that this is not the case for the tweet-news category, where the text is more ``free form'' and less normative and one would expect sublexical approaches to have stronger performance.

%%% Local Variables: 
%%% mode: latex
%%% TeX-master: "sts14-ntnu"
%%% End: 


\section{Conclusion and Future Work}


The NTNU system can be regarded as continuation of the most successful systems from the STS'12 shared task,  combining shallow textual, distributional and knowledge-based features into a support vector regression model. It reuses features from the TakeLab and DKPro systems, resulting in a very strong baseline. 

Adding new features to further improve performance turned out to be hard: 
only \texttt{GateWordMatch} yielded improved performance. 
Similarity features based on both classical and innovative variants of Random Indexing were shown to correlate with semantic textual similarity, but did not complement the existing distributional features. Likewise, features designed to reveal deeper syntactic (graph edit distance) and semantic relations (RelEx) did not add to the score. 

As future work, we would aim to explore a vertical feature composition approach similar 
to \texttt{GateWordMatch} and contrast it with the ``flat'' composition currently used in our systems. 

\section*{Acknowledgements}

Thanks to TakeLab for source code of their `\emph{simple\/}' system and the full-scale LSA models. 
Thanks to the team from Ubiquitous Knowledge Processing Lab for source code of their DKPro Similarity system.

\vfill
\pagebreak
\balance
%\bibliographystyle{naaclhlt2013} %EM: official bibtex style seems not to work...
\bibliographystyle{apa}

\bibliography{sts13-ntnu-core}

\end{document}
