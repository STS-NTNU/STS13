\documentclass[11pt,letterpaper]{article}
\usepackage{naaclhlt2013}
\usepackage{times}
\usepackage{latexsym}
\usepackage{url}
\usepackage{todonotes}
\setlength\titlebox{6.5cm}    % Expanding the titlebox

\title{NTNU-CORE: title}

\author{Authors\\
  Norwegian University of Science and Technology\\
  Department of Computer and Information and Science\\
  Sem S{\ae}lands vei 7-9 \\
  NO-7491 Trondheim, Norway\\
  {\tt author1@idi.ntnu.no}}

\date{}

\begin{document}
\maketitle
\begin{abstract}
ABSTRACT
\end{abstract}

\section{Introduction}



\section{Reused features}

The TakeLab system for measuring semantic text similarity
\cite{vsaric2012takelab} was one of the most succesful systems in the
STS 2012 shared task. There were two variants called the \emph{simple}
and \emph{syntax} system. The \emph{simple} system obtained 3rd place
for overall Pearson and 1st for normalized Pearson. We used the source
code\footnote{\url{http://takelab.fer.hr/sts/}} to generate the
features for the STS 2012 data as well as the STS 2013 test data. The
later required LSA vector space models, which were kindly provided by
the TakeLab team, and word counts, which were obtained from Google
Books Ngrams (version 20120701, 468,491,999,592
words)\footnote{\url{http://storage.googleapis.com/books/ngrams/books/datasetsv2.html}}.

Since the TakeLab features are described in\cite{vsaric2012takelab},
they will be reviewed only briefly here. The\emph{ n-gram overlap
  features} measure overlap in unigrams, bigrams and trigrams of
lower-cased words and lemmas, filtering stopwords and non-words (we
did not use content n-gram overlap or skip n-grams). The\emph{
  WordNet-augmented word overlap feature} measures unigram overlap
where the similarity between word pairs is defined as the harmonic
mean of their WordNet path length similarities \cite{}. The weighted
word overlap features measure unigram overlap for lower-cased words
and lemmas, where each word is weighted according to its Information
Content calculated on the basis of unigram counts from Google Books
Ngrams. The \emph{vector space sentence similarity features} capture
distributional similarity between words of both sentences. Vector
space models are derived from two corpora -- the New York Times
Annotated Corpus (NYT) and Wikipedia (wiki) -- using Latent Semantic
Analysis \cite{DeerwesterDumaisFurnas:1990}. A sentence vector is
obtained by summing the vector of each word in the sentence.  A
weighted variant uses IC to weight the vector of each word before
summation. Vector space sentence similarity is then computed as the
cosine similarity between the sentence vectors. The \emph{normalized
  difference} features measure measure the normalized differences in
sentence length (in lower-cased stopword-filtered words) and in
aggregated information content (sum of IC over all lower-cased
words). The \emph{shallow NE similarity feature} expresses unigram
overlap in named entities, treating each capitalized word longer than
one character as a named entity (excluding the first word in the
sentence). Finally, the\emph{ numbers overlap} features and the
\emph{stock index} features measure the overlap and count of numerals
and stock index symbols respectively. 
% EM: provide more details if space allows

DKPro system \cite{bar2012ukp}


\section{Compositional Word Matching}
\label{gleb-feats}

Compositional word matching similarity is based on a one-to-one alignment of words from the two sentences. The alignment is obtained by maximal weighted bipartite matching using several word similarity measures. In addition, we utilise named entity recognition and matching tools. In general, the approach is similar to the one described by \citet{Karnick2012}, with a different set of tools used. 
Our implementation relies on the ANNIE components in GATE~\citep{CunninghamEA:02} 
and will thus be referred to as \feat{GateWordMatch}.

The processing pipeline for \feat{GateWordMatch} is:  
(1)~tokenization by ANNIE English Tokeniser, 
(2)~part-of-speech tagging by ANNIE POS Tagger, 
(3)~lemmatization by GATE Morphological Analyser, 
(4)~stopword removal, 
(5)~named entity recognition based on lists by ANNIE Gazetteer, 
(6)~named entity recognition based on the JAPE grammar by the ANNIE NE Transducer, 
(7)~matching of named entities by ANNIE Ortho Matcher, 
(8)~computing WordNet and Levenstein similarity between words, 
(9)~calculation of a maximum weighted bipartite graph matching based on similarities from 7~and~8. 

Steps 1--4 are standard preprocessing routines. 
In step~5, named entities are recognised based on lists that contain locations, organisations, companies, newspapers, and person names, as well as date, time and currency units. 
In step~6, JAPE grammar rules are applied to recognise entities such as addresses, emails, dates, job titles, and person names based on basic syntactic and morphological features. 
Matching of named entities in step~7 is based on matching rules that check the type of named entity, and lists with aliases to identify entities as ``US'', ``United State'', and ``USA'' as the same entity. 

In step~8, similarity is computed for each pair of words from the two sentences. Words that are matched as entities in step~7 get a similarity value of $1.0$. 
For the rest of the entities and non-entity words we use \emph{LCH\/} \citep{Leacock1998} similarity, which is based on a shortest path between the corresponding senses in WordNet. Since word sense disambiguation is not used, we take the similarity between the nearest senses of two words.
%The \emph{LCH\/} measure is limited to nouns and verbs and does not support cross-POS (part of speech) similarity. 
For cases when the WordNet-based similarity cannot be obtained, a similarity based on the Levenshtein distance \citep{Levenshtein:66} is used instead. It is normalised by the length of the longest word in the pair. For the STS'13 test data set, named entity matching contributed to 4\% of all matched word pairs; LCH similarity to 61\%, and Levenshtein distance to 35\%.

In step~9, 
%a complete bipartite graph is constructed and the 
maximum weighted bipartite matching is computed using the Hungarian Algorithm \citep{Kuhn1955}. 
Nodes in the bipartite graph represent words from the sentences, 
and edges have weights that correspond to similarities between tokens obtained in step~8.
Weighted bipartite matching finds the one-to-one alignment that maximizes 
%the total similarity of the matching, which is 
the sum of similarities between aligned tokens. 
Total similarity normalised by the number of words in both sentences 
is used as the final sentence similarity measure.


\section{Hans feats}




%%% Local Variables: 
%%% mode: latex
%%% TeX-master: "sts13-ntnu-core"
%%% End: 



\section{Deeper Semantic Relations}
\label{lars-feats}

Two deep strategies were employed to accompany the shallow-processed feature sets.  
Two existing systems were used to provide the basis for these features, namely the 
RelEx system~\citep{bioinflmu-254} from the OpenCog initiative~\citep{Hart:2008:OSF:1566174.1566223}, 
and an in-house graph-edit distance system
developed for plagiarism detection~\citep{rokenes13}.

%\subsection{RelEx}

RelEx outputs syntactic trees, dependency graphs, and \emph{semantic frames\/} as this one
for the sentence \linebreak
``\emph{Indian air force to buy 126 Rafale fighter jets\/}'':
\vspace*{-2.5ex}
%\begin{quote}
{\footnotesize\texttt{
\begin{mdframed}[linewidth=2pt]
Commerce\_buy:Goods(buy,jet)\\
Entity:Entity(jet,jet)\\
Entity:Name(jet,Rafale)\\
Entity:Name(jet,fighter)\\
Possibilities:Event(hyp,buy)\\
Request:Addressee(air,you)\\
Request:Message(air,air)\\
Transitive\_action:Beneficiary(buy,jet)
\end{mdframed}}}
%\end{quote}
\vspace*{2.5ex}
\noindent
Three features were extracted from this:
first, if there was an exact match of the frame found in $s_1$ with $s_2$;
second, if there was a partial match until the first argument (\texttt{Commerce\_buy:Goods(buy});
and third if there was a match of the frame category (\texttt{Commerce\_buy:Goods}).

%\subsection{Graph-edit distance}

In STS'12, \citet{singh-bhattacharya-bhattacharyya:2012:STARSEM-SEMEVAL} 
matched Universal Networking Language (UNL) graphs against each other 
by counting matches of relations and universal words, while
\citet{bhagwani-satapathy-karnick:2012:STARSEM-SEMEVAL} calculated
WordNet-based word-level similarities and created a weighted bipartite graph 
(see Section~\ref{gleb-feats}).
%
The method employed here instead looked at the graph edit distance 
between dependency graphs obtained with the
Maltparser dependency parser \citep{Nivre06maltparser:a}.
Edit distance is the defined as the minimum of the sum of the costs of the edit operations (insertion, deletion and substitution of nodes) required to transform one graph into the other.
It is approximated with a fast but suboptimal algorithm based on bipartite graph matching through the Hungarian algorithm~\citep{RiesenBunke:2009}.
%Edit cost was calculated between two dependency graphs obtained with.  
%The nodes in the two graphs were then assigned to each other using the Hungarian algorithm.  
%The cost was determined by looking at the number of additions, deletions and substitutions needed to transform one node to the other (having the same dependencies).



\section{Results}


\section{Discussion}


\section{Conlusion}


\section*{Acknowledgements}

Thanks to the TakeLab team from the University of Zagreb for making
the code of the \emph{simple} system publicly available as well as
providing us with the full-scale LSA models. Thanks to the team from
Ubiquitous Knowledge Processing Lab in Darmstadt for releasing the
code of the DKPro Similarity system.


\bibliographystyle{naaclhlt2013.bst}
\bibliography{sts13-ntnu-core}

\end{document}
