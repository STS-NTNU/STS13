\section{Deeper Semantic Relations}
\label{lars-feats}

Two deep strategies were employed to accompany the shallow-processed feature sets.  
Two existing systems were used to provide the basis for these features, namely the 
RelEx system~\citep{bioinflmu-254} from the OpenCog initiative~\citep{Hart:2008:OSF:1566174.1566223}, 
and an in-house graph-edit distance system
developed for plagiarism detection~\citep{rokenes13}.

%\subsection{RelEx}

RelEx outputs syntactic trees, dependency graphs, and \emph{semantic frames\/} as this one
for the sentence \linebreak
``\emph{Indian air force to buy 126 Rafale fighter jets\/}'':
\vspace*{-2.5ex}
%\begin{quote}
{\footnotesize\texttt{
\begin{mdframed}[linewidth=2pt]
Commerce\_buy:Goods(buy,jet)\\
Entity:Entity(jet,jet)\\
Entity:Name(jet,Rafale)\\
Entity:Name(jet,fighter)\\
Possibilities:Event(hyp,buy)\\
Request:Addressee(air,you)\\
Request:Message(air,air)\\
Transitive\_action:Beneficiary(buy,jet)
\end{mdframed}}}
%\end{quote}
\vspace*{2.5ex}
\noindent
Three features were extracted from this:
first, if there was an exact match of the frame found in $s_1$ with $s_2$;
second, if there was a partial match until the first argument (\texttt{Commerce\_buy:Goods(buy});
and third if there was a match of the frame category (\texttt{Commerce\_buy:Goods}).

%\subsection{Graph-edit distance}

In STS'12, \citet{singh-bhattacharya-bhattacharyya:2012:STARSEM-SEMEVAL} 
matched Universal Networking Language (UNL) graphs against each other 
by counting matches of relations and universal words, while
\citet{bhagwani-satapathy-karnick:2012:STARSEM-SEMEVAL} calculated
WordNet-based word-level similarities and created a weighted bipartite graph 
(see Section~\ref{gleb-feats}).
%
The method employed here instead looked at the graph edit distance 
between dependency graphs obtained with the
Maltparser dependency parser \citep{Nivre06maltparser:a}.
Edit distance is the defined as the minimum of the sum of the costs of the edit operations (insertion, deletion and substitution of nodes) required to transform one graph into the other.
It is approximated with a fast but suboptimal algorithm based on bipartite graph matching through the Hungarian algorithm~\citep{RiesenBunke:2009}.
%Edit cost was calculated between two dependency graphs obtained with.  
%The nodes in the two graphs were then assigned to each other using the Hungarian algorithm.  
%The cost was determined by looking at the number of additions, deletions and substitutions needed to transform one node to the other (having the same dependencies).
