\section{Lars feats}

Two deep strategies were employed to accompany shallow-processed feature sets.  Two systems were used to provide the basis for such feature, namely the Relex system \cite{bioinflmu-254} from the OpenCog initiative \cite{Hart:2008:OSF:1566174.1566223}, and a graph-based edit-distance system from a recent M.Sc thesis \cite{rokenes13}.

Relex outputs semantic relations found in sentences 

\subsection{Relex}

Relex outputs syntatic trees, dependency parses as well as semantic \emph{frames} exemplified (in part) below for the sentence \emph{Indian air force to buy 126 Rafale fighter jets}
 
\begin{list}{}{}
\item Commerce\_buy:Goods(buy,jet)
\item Entity:Entity(jet,jet)
\item Entity:Name(jet,Rafale)
\item Entity:Name(jet,fighter)
\item Possibilities:Event(hyp,buy)
\item Request:Addressee(air,you)
\item Request:Message(air,air)
\item Transitive\_action:Beneficiary(buy,jet)
\end{list}



Based on this information, three features were extracted.  First, if there was as full (exact) match of the frame found in S1 in S2.  Second if there was a partial match until the first argument (i.e \emph{Categorization:Category(jet} and third if there was a match of the frame category (i.e \emph{Entity:Name}).

\subsection{Graph-edit distance}

The second deep feature was extracted by using the above-mentioned system for extracting the graph-edit distance betweeen the dependency (Maltparser)  output of the two sentences.  Both absolute and normalized scores were output by the system.

%%% Local Variables: 
%%% mode: latex
%%% TeX-master: "sts13-ntnu-core"
%%% End: 
